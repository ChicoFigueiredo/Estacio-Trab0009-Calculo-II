% Comprensão
\pdfminorversion=5
\pdfcompresslevel=9
\pdfobjcompresslevel=2

\documentclass[
	12pt,				% tamanho da fonte
	openright,			% capítulos começam em pág ímpar (insere página vazia caso preciso)
	%twoside,			% para impressão em recto e verso. Oposto a oneside
	oneside,
	a4paper,			% tamanho do papel.
	chapter=TITLE,		% títulos de capítulos convertidos em letras maiúsculas
	section=TITLE,		% títulos de seções convertidos em letras maiúsculas
	sumario=abnt-6027-2012,
	english,			% idioma adicional para hifenização
	brazil				% o último idioma é o principal do documento
]{abntex2}

\label{oculto:packages}
% ----------------------------------------------------------
% Pacotes básicos
% ----------------------------------------------------------
\usepackage{import}                            % package import tem o comando import que faz a importação com "novo path"
\usepackage{amsmath}							% Pacote matemático
\usepackage{amssymb}							% Pacote matemático
\usepackage{amsfonts}							% Pacote matemático
\usepackage{lmodern}							% Usa a fonte Latin Modern
\usepackage[T1]{fontenc}						% Selecao de codigos de fonte.
\usepackage[utf8]{inputenc}						% Codificacao do documento (conversão automática dos acentos)
\usepackage{lastpage}							% Usado pela Ficha catalográfica
\usepackage{indentfirst}						% Indenta o primeiro parágrafo de cada seção.
\usepackage[dvipsnames,svgnames,table]{xcolor}			% Controle das cores
\usepackage{graphicx}							% Inclusão de gráficos
\usepackage{microtype} 							% para melhorias de justificação
\usepackage{lipsum}								% para geração de dummy text
\usepackage[brazilian,hyperpageref]{backref}	% Paginas com as citações na bibl
%\usepackage[alf]{abntex2cite}					% Citações padrão ABNT
\usepackage[num]{abntex2cite}					% Citações padrão ABNT numérica
\usepackage{adjustbox}							% Pacote de ajuste de boxes
\usepackage{subcaption}							% Inclusão de Subfiguras e sublegendas
\usepackage{enumitem}							% Personalização de listas
\usepackage{siunitx}							% Grandezas e unidades
\usepackage[section]{placeins}					% Manter as figuras delimitadas na respectiva seção com a opção [section]
\usepackage{multirow}							% Multi colunas nas tabelas
\usepackage{array,tabularx} 					% Pacotes de tabelas
\usepackage{booktabs}							% Pacote de tabela profissonal
\usepackage{rotating}							% Rotacionar figuras e tabelas
\usepackage{xfrac}								% Fazer frações n/d em linha
\usepackage{bm}									% Negrito em modo matemático
\usepackage{xstring}							% Manipulação de strings
\usepackage{pgfplots}							% Pacote de Gráficos
\usepackage{tikz}								% Pacote de Figuras
\usepackage[american, cuteinductors,smartlabels, fulldiode, siunitx, americanvoltages, oldvoltagedirection, smartlabels]{circuitikz}						% Pacote de circuitos elétricos
\usepackage{lipsum}
\usepackage{xargs}

%\usepackage[style=abnt]{biblatex}

\usepackage{hyperref}
% informações do PDF
\makeatletter
	\hypersetup{
		%pagebackref=true,
		pdftitle={\@title},
		pdfauthor={\@author},
		pdfsubject={\imprimirpreambulo},
		pdfcreator={LaTeX with abnTeX2},
		pdfkeywords={abnt}{latex}{abntex}{abntex2}{trabalho academico},
		colorlinks=true,       		% false: boxed links; true: colored links
		linkcolor=NavyBlue,          	% color of internal links
		citecolor=NavyBlue,        	% color of links to bibliography
		filecolor=black,      		% color of file links
		urlcolor=NavyBlue,
		bookmarksdepth=4,
		linktoc=all
	}
\makeatother


% Centraliza captions of pictures
\usepackage[justification=centering]{caption}


% glossário, com fix para erros
%\usepackage[acronym]{glossaries}


% To use externalize consider
%https://tex.stackexchange.com/questions/182783/tikzexternalize-not-compatible-with-miktex-2-9-abntex2-package
%Lauro Cesar digged into the problem until he came with a solution for me to test. And it Works!
%
%According to this link:
%
%The package calc changed the commands \setcounter and friends to be fragile. So you have to make them robust. The example below uses etoolbox with \robustify:
%
\usepackage{etoolbox}
\robustify\setcounter
\robustify\addtocounter
\robustify\setlength
\robustify\addtolength

\usepackage[]{tocloft}
%\setlength\cftsectionnumwidth{4em}
%\setlength{\cftchapterindent}{2em}
%\setlength{\cftsectionindent}{5em}
%\setlength{\cftsubsectionindent}{8em}

\usepackage{pstricks-add}
\pgfplotsset{compat=1.15}
\usepackage{mathrsfs}
\usetikzlibrary{arrows}

%<<<<<<<WARNING>>>>>>>
% PGF/Tikz doesn't support the following mathematical functions:
% cosh, acosh, sinh, asinh, tanh, atanh,
% x^r with r not integer

% Plotting will be done using GNUPLOT
% GNUPLOT must be installed and you must allow Latex to call external
% programs by adding the following option to your compiler
% shell-escape    OR    enable-write18 
% Example: pdflatex --shell-escape file.tex 


%% How to silence memoir class warning against the use of caption package?
%% https://tex.stackexchange.com/questions/391993/how-to-silence-memoir-class-warning-against-the-use-of-caption-package
%\usepackage{silence}
%\WarningFilter*{memoir}{You are using the caption package with the memoir class}
%\WarningFilter*{Class memoir Warning}{You are using the caption package with the memoir class}


% -----------------------------------------------------------------
% Você pode adicionar seus pacotes a partir desta linha;
% -----------------------------------------------------------------

%\usepackage[showframe,pass]{geometry}
%\usepackage[11,12]{pagesel}
\usepackage{qrcode}
\usepackage{multirow}

% -----------------------------------------------------------------
% pacote interno
\usepackage{modulos/trabalhos.academicos}



% CONFIGURAÇÕES DE PACOTES
% Configurações do pacote backref
% Usado sem a opção hyperpageref de backref
\renewcommand{\backrefpagesname}{Citado na(s) página(s):~}

% Texto padrão antes do número das páginas
\renewcommand{\backref}{}

% Define os textos da citação
\renewcommand*{\backrefalt}[4]{
	\ifcase #1 %
		Nenhuma citação no texto.%
	\or
		Citado na página #2.%
	\else
		Citado #1 vezes nas páginas #2.%
	\fi
}%

% Citação online --- MODIFICAR ---
\newcommandx{\citeaa}[2][2= ]{\citeauthoronline{#1}~(\citeyear{#1}#2)}


\label{oculto-programinhas}
%-----------------------------------------
% (1) simple command for print or not
%-----------------------------------------
\usepackage{ifthen}
\newcommand{\ImprimirSimOuNao}[2][Sim]
{
  \ifthenelse{\equal{#1}{Sim}}{#2}{}
}

% alterando o aspecto da cor azul
%\definecolor{blue}{RGB}{41,5,195}

\makeatletter
    \newcommand{\includetikz}[1]{%
    	\tikzsetnextfilename{#1}%
    	\input{#1.tex}%
    }
\makeatother


\newcommand{\MONTH}{%
	\ifcase\the\month
	\or JAN% 1
	\or FEV% 2
	\or MAR% 3
	\or ABR% 4
	\or MAI% 5
	\or JUN% 6
	\or JUL% 7
	\or AUG% 8
	\or SET% 9
	\or OUT% 10
	\or NOV% 11
	\or DEZ% 12
	\fi}
\makeatletter

% Personalização das opções das listas
\setlist[itemize]{leftmargin=\parindent}

\newcommand{\me}[1]{Elaborado pelo autor, #1.}

% Configuração do pgfplots
\pgfplotsset{compat=newest} %compat=1.14
\pgfplotsset{plot coordinates/math parser=false}
\newlength\figureheight
\newlength\figurewidth

% Libraries do TiKz
\usetikzlibrary{quotes,angles,arrows}
\usetikzlibrary{through,calc,math}
\usetikzlibrary{graphs,backgrounds,fit}
\usetikzlibrary{shapes,positioning,patterns,shadows}
\usetikzlibrary{decorations.pathreplacing}
\usetikzlibrary{shapes.geometric}
\usetikzlibrary{arrows.meta}
\usetikzlibrary{external}

%\tikzexternalize[]
%\tikzexternalenable
%\tikzexternalize
%\tikzexternaldisable
%\tikzset{external/force remake}
%\tikzexternalize[shell escape=-enable-write18]

% Configurações do CircuiTiKz
\ctikzset{bipoles/thickness=1}
%\ctikzset{bipoles/length=1.2cm}
\ctikzset{monopoles/ground/width/.initial=.2}
\ctikzset{bipoles/resistor/height=0.25}
\ctikzset{bipoles/resistor/width=0.6}
\ctikzset{bipoles/capacitor/height=0.5}
\ctikzset{bipoles/capacitor/width=0.15}
\ctikzset{bipoles/generic/height=0.25}
\ctikzset{bipoles/generic/width=0.6}
%\ctikzset{bipoles/capacitor polar/length=0.5}
%\ctikzset{bipoles/diode/height=.375}
%\ctikzset{bipoles/diode/width=.3}
%\ctikzset{tripoles/thyristor/height=.8}
%\ctikzset{tripoles/thyristor/width=1}
\ctikzset{bipoles/vsourcesin/height=.5}
\ctikzset{bipoles/vsourcesin/width=.5}
\ctikzset{bipoles/cvsourceam/height=.6}
\ctikzset{bipoles/cvsourceam/width=.6}
%\ctikzset{tripoles/european controlled voltage source/width=.4}

\tikzstyle{every node}=[font=\footnotesize]
\tikzstyle{every path}=[line width=0.25pt,line cap=round,line join=round]
%\tikzstyle{every path}=[line cap=round,line join=round]


% Definição de cores MATLAB
\definecolor{matlab_blue}{rgb}	{         0,    0.4470,    0.7410}
\definecolor{matlab_orange}{rgb}{    0.8500,    0.3250,    0.0980}
\definecolor{matlab_yellow}{rgb}{    0.9290,    0.6940,    0.1250}
\definecolor{matlab_violet}{rgb}{    0.4940,    0.1840,    0.5560}
\definecolor{matlab_green}{rgb}	{	 0.4660,    0.6740,    0.1880}
\definecolor{matlab_lblue}{rgb}	{    0.3010,    0.7450,    0.9330}
\definecolor{matlab_red}{rgb}	{    0.6350,    0.0780,    0.1840}

% Personalização das legendas
\usepackage[format = hang,
			labelsep = endash,
			singlelinecheck = false,
			skip = 6pt,
			listformat = simple]{caption}

% Personalização das unidades
\sisetup{output-decimal-marker = {,}}
\sisetup{exponent-product = \cdot, output-product = \cdot}
\sisetup{tight-spacing=true}
\sisetup{group-digits = false}

% Personalizações de tipo de colunas de tabelas
\newcolumntype{L}[1]{>{\raggedright\let\newline\\\arraybackslash\hspace{0pt}}m{#1}}
\newcolumntype{C}[1]{>{\centering\let\newline\\\arraybackslash\hspace{0pt}}m{#1}}
\newcolumntype{R}[1]{>{\raggedleft\let\newline\\\arraybackslash\hspace{0pt}}m{#1}}

% Personalizações de cores da UDESC
\definecolor{CapaAmareloUDESC}{RGB}{243,186,83}		% Especializacao
\definecolor{CapaVerdeUDESC}{RGB}{0,112,52}			% Mestrado
\definecolor{CapaVermelhoUDESC}{RGB}{171,35,21}		% Doutorado
\definecolor{CapaAzulUDESC}{RGB}{38,54,118} 		% Pós-Doutorado


% Espaçamento depois do título
\setlength{\afterchapskip}{0.7\baselineskip}
% O tamanho do parágrafo é dado por:
\setlength{\parindent}{1.25cm}
% Controle do espaçamento entre um parágrafo e outro:
\setlength{\parskip}{0.15cm}  % tente também \onelineskip
%\SingleSpacing % Espaçamento simples
\OnehalfSpacing % Espaçamento 1,5 (UDESC/CCT)
%\DoubleSpacing	% Espaçamento duplo

% ---
% Margens - NBR 14724/2011 - 5.1 Formato
% ---
\setlrmarginsandblock{3cm}{2cm}{*}
\setulmarginsandblock{3cm}{2cm}{*}
\checkandfixthelayout[fixed]
% ---









\label{oculto:dados-do-trabalho}
% -----------------------------------------------------------------
% Informações de dados para CAPA e FOLHA DE ROSTO
% -----------------------------------------------------------------
\tipotrabalho{
    Atividade Estruturada
}

\titulo{
	Resenha da importância do conteúdo de Calculo I na formação de professores
}%

% ATENÇÃO: O símbolo {} indica o sobrenome para a ficha catalográfica.
%          Exemplo: Sherlock Holmes {}da Silva para sobrenomes compostos;
%          Exemplo: Arnold Alois {}Schwarzenegger para sobrenome simples.
\autor{Francisco Lima {}Figueiredo}%
\newcommand\matricula{Matricula 202201097825}
\orientador{Andre Luis {}Funcke}%
%\coorientador{}%
\instituicao{Universidade Estácio de Sá}%

\preambulo{
	Trabalho apresentado ao professor Andre Luis Funcke como requisito parcial para obtenção da aprovação na disciplina 
    Cálculo I (CEL1397) 9001.
}

\local{Brasília}%
\data{\MONTH/\the\year}%
% ---










% compila o indice
\makeindex

% -----------------------------------------------------------------
% Início do documento
% -----------------------------------------------------------------
\begin{document}
	
	\selectlanguage{brazil}
	\frenchspacing            % Retira espaço extra obsoleto entre as frases.
	
	% Você pode comentar os elementos que não deseja em seu trabalho;

	% -----------------------------------------------------------------
	% ELEMENTOS PRÉ-TEXTUAIS
	% -----------------------------------------------------------------
	\pretextual

	\label{oculto:capa}
	% --- 
	% Capa
	% --- 
	%\imprimircapaTrabalho      % Capa UDESC para Trabalho
	\imprimircapaTCC			% Capa UDESC para TCC
	%\imprimircapaDissertacao	% Capa UDESC para Dissertações
	%\imprimircapaTese			% Capa UDESC para Teses
	%\imprimircapaPosDoc		% Capa UDESC para Pós-Doutorado

	%\imprimirCapaCustom{CapaAmareloUDESC}{\imprimirtipotrabalho}{\imprimirinstituicao}  %Capa UDESC para Pós-Doutorado

	%\imprimircapa				% Capa padrão


	% -----------------------------------------------------------------
	% ELEMENTOS PRÉ-TEXTUAIS
	% -----------------------------------------------------------------

		% ---
		% Folha de rosto
		% (o * indica que haverá a ficha bibliográfica)
		% ---
		\ImprimirSimOuNao[Sim]{             % Elemento Obrigatório
			\imprimirfolhaderosto*                                                 
		}
		% ---




		\label{oculto:ficha-bibliográfica}
		% ---
		% Inserir a ficha bibliografica
		% ---
		\ImprimirSimOuNao[Não]{             % Elemento Obrigatório

			% Isto é um exemplo de Ficha Catalográfica, ou ``Dados internacionais de
			% catalogação-na-publicação''. Você pode utilizar este modelo como referência.
			% Porém, provavelmente a biblioteca da sua universidade lhe fornecerá um PDF
			% com a ficha catalográfica definitiva após a defesa do trabalho. Quando estiver
			% com o documento, salve-o como PDF no diretório do seu projeto e substitua todo
			% o conteúdo de implementação deste arquivo pelo comando abaixo:

			% \begin{fichacatalografica}
			%     \includepdf{fig_ficha_catalografica.pdf}
			% \end{fichacatalografica}

			%	\setlength{\parindent}{0cm}
			%	\setlength{\parskip}{0pt}
			\begin{fichacatalografica}                            
				%\sffamily
				%\rmfamily
				\ttfamily \hbadness=10000
				\vspace*{\fill}					% Posição vertical
				\begin{center}					% Minipage Centralizado
			%	\begin{minipage}[c]{8cm}
			%	\centering \sffamily
			%	 Ficha catalográfica elaborada pelo(a) autor(a), com auxílio do programa de geração automática da Biblioteca Setorial do CCT/UDESC
			%	\end{minipage}
				\fbox{\begin{minipage}[c]{12.5cm}		% Largura
				\flushright
				{\begin{minipage}[c]{10.5cm}		% Largura
				\vspace{1.25cm}
				%\footnotesize
				\setlength{\parindent}{1.5em}
				\noindent \invertname{\imprimirautor} \par
				\imprimirtitulo{ }/{ }\imprimirautor. -- \imprimirlocal, \imprimirdata .\par
				\pageref{LastPage} p. : il. ; 30 cm.\par
				\vspace{1.5em}
				\imprimirorientadorRotulo~\imprimirorientador\par
				\imprimircoorientadorRotulo~\imprimircoorientador\par
				\imprimirtipotrabalho~--~\imprimirinstituicao, \imprimirlocal, \imprimirdata.\par
				\vspace{1.5em}
					1. Sociologia.
					2. Meio Ambiente.
					I. \invertname{\imprimirorientador}.
					II. \invertname{\imprimircoorientador}.
					III. \imprimirinstituicao. %
				\vspace{1.25cm}	%
				\end{minipage}%
				}%
				\end{minipage}}%
				\end{center}
			\end{fichacatalografica}
		}




		% ---
		% Inserir errata
		% ---
		\ImprimirSimOuNao[Não]{
			\begin{errata}
				Elemento opcional da \citeonline[4.2.1.2]{NBR14724:2011}. Exemplo:

				\vspace{\onelineskip}

				FERRIGNO, C. R. A. \textbf{Tratamento de neoplasias ósseas apendiculares com
				reimplantação de enxerto ósseo autólogo autoclavado associado ao plasma
				rico em plaquetas}: estudo crítico na cirurgia de preservação de membro em
				cães. 2011. 128 f. Tese (Livre-Docência) - Faculdade de Medicina Veterinária e
				Zootecnia, Universidade de São Paulo, São Paulo, 2011.

				\begin{table}[htb]
				\center
				\footnotesize
				\begin{tabular}{|p{1.4cm}|p{1cm}|p{3cm}|p{3cm}|}
				  \hline
				   \textbf{Folha} & \textbf{Linha}  & \textbf{Onde se lê}  & \textbf{Leia-se}  \\
					\hline
					1 & 10 & auto-conclavo & autoconclavo\\
				   \hline
				\end{tabular}
				\end{table}
				
			\end{errata}
			% ---
		}




		
		% ---
		% Inserir folha de aprovação
		% ---
		\ImprimirSimOuNao[Não]{
			\begin{folhadeaprovacao}
				\begin{center}
					{\MakeTextUppercase{\ABNTEXchapterfont\large\imprimirautor}}

					\vspace*{\fill} %\vspace*{\fill}
					\begin{center}
					  {\MakeTextUppercase{\ABNTEXchapterfont\bfseries\large\imprimirtitulo}}
					\end{center}
					\vspace*{\fill}
					
					%\hspace{.45\textwidth}
					{\begin{minipage}[c]{1\linewidth}
						\setlength{\parindent}{1.25cm}
						\imprimirpreambulo
					\end{minipage}}%
					\vspace*{\fill}
					\end{center}
						
					 
					{\bfseries Banca Examinadora: }
					\vspace*{\fill}
					
					{Orientador: \vspace{-16pt} }
					\assinatura{\textbf{Prof. \imprimirorientador , Dr.} \\ Univ. XXX} 
					{Coorientador: \vspace{-16pt}}   
					\assinatura{\textbf{Prof. \imprimircoorientador , Dr.} \\ Univ. XXX}
				   
					{Membros: \vspace{-16pt} } 
					
				% --- Exemplo de assinaturas em sequência ---       
					\setlength{\ABNTEXsignwidth}{8.5cm}
					
					\assinatura{\textbf{Prof. Professor, Dr.} \\ Univ. XXX}
					\assinatura{\textbf{Prof. Professor, Dr.} \\ Univ. XXX}
					\assinatura{\textbf{Prof. Professor, Dr.} \\ Univ. XXX}

				% --- Exemplo de assinaturas lado a lado ---
					\setlength{\ABNTEXsignwidth}{7.5cm}
				%
				%    \noindent\hfill\assinatura*{\textbf{Prof. Professor, Dr.} \\ Univ. XXX}%
				%    \hfill%
				%    \assinatura*{\textbf{Prof. Professor, Dr.} \\ Univ. XXX}%
				%    \hfill
				%    
				%    \noindent\hfill\assinatura*{\textbf{Prof. Professor, Dr.} \\ Univ. XXX}%
				%    \hfill%
				%    \assinatura*{\textbf{Prof. Professor, Dr.} \\ Univ. XXX}%
				%    \hfill
					
					\vspace*{\fill}  
					\begin{center}
					{\large\imprimirlocal, 01 de maio de \imprimirdata}
				\end{center}
				\vspace*{1cm}  
			\end{folhadeaprovacao}
		}
		% ---




		\label{oculto:dedicatória}
		% ---
		% Dedicatória
		% ---
		\ImprimirSimOuNao[Não]{
			\begin{dedicatoria}
			   \vspace*{\fill}
			   \centering
			   \noindent
			   \textit{ Este trabalho é dedicado às minhas maiores inspirações,\\
			   Minha esposa Daiane, minha filha Mayara Barros e minha sobrinha do coração Rachael Costa.} \vspace*{\fill}
			\end{dedicatoria}		
		}
		% ---
		
		
		
		
		\label{oculto:agradecimentos}
		% ---
		% Agradecimentos
		% ---
		\ImprimirSimOuNao[Não]{
			\begin{agradecimentos}
				Os agradecimentos principais são direcionados à Gerald Weber, Miguel Frasson, Leslie H. Watter, Bruno Parente Lima, Flávio de Vasconcellos Corrêa, Otavio Real Salvador, Renato Machnievscz\footnote{Os nomes dos integrantes do primeiro projeto abn\TeX\ foram extraídos de \url{http://codigolivre.org.br/projects/abntex/}} e todos aqueles que contribuíram para que a produção de trabalhos acadêmicos conforme as normas ABNT com \LaTeX\ fosse possível.

				Agradecimentos especiais são direcionados ao Centro de Pesquisa em Arquitetura da Informação\footnote{\url{http://www.cpai.unb.br/}} da Universidade de Brasília (CPAI), ao grupo de usuários \emph{latex-br}\footnote{\url{http://groups.google.com/group/latex-br}} e aos novos voluntários do grupo \emph{\abnTeX}\footnote{\url{http://groups.google.com/group/abntex2} e \url{http://www.abntex.net.br/}}~que contribuíram e que ainda contribuirão para a evolução do \abnTeX.
			\end{agradecimentos}
		}
		% ---




		\label{oculto:epigrafe}
		% ---
		% Epígrafe
		% ---
		\ImprimirSimOuNao[Sim]{
			\begin{epigrafe}
				\vspace*{\fill}
				\hspace{.35\textwidth}
				{\begin{minipage}{.6\textwidth}
					\begin{flushright}
						\textit{``Porque eu fazia do amor um cálculo matemático errado: pensava que, somando as compreensões, eu amava. Não sabia que, somando as incompreensões é que se ama verdadeiramente.``\\
							(Clarice Lispector)}
					\end{flushright}
				\end{minipage}}%
			\end{epigrafe}
		}
		% ---
		
		
		
		
		\label{oculto:resumo}
		% ---
		% RESUMOS
		% ---
		\ImprimirSimOuNao[Não]{
			% resumo em português
			\setlength{\absparsep}{18pt} % ajusta o espaçamento dos parágrafos do resumo
			\begin{resumo}
			 O presente estudo tem como objetivo coletar diversas estatísticas, parâmetros e métricas de provas do ENEM de vários anos e discorrer sobre a prova, o desempenho na disciplina de matemática e os eventuais fatores e eventos correlatos que possam indicar causa-efeito que afetem a nota final da disciplina na avaliação nacional. \\
			 Foram analisadas provas entre os anos de 2015 a 2019 do ENEM, seus gabaritos e os microdados de cada aluno, o que fornece um ambiente rico para perguntas e testes de hipóteses.\\
			 

			 \textbf{Palavras-chave}: ENEM, estatísticas, habilidades, competências, métricas, parâmetros.
			\end{resumo}
		}
		
		
		
		
		\label{oculto:abstract}
		% ---
		% Abstract
		% ---
		\ImprimirSimOuNao[Não]{
			% resumo em inglês
			\begin{resumo}[Abstract]
			 \begin{otherlanguage*}{english}
			   This is the english abstract.

			   \vspace{\onelineskip}
			 
			   \noindent 
			   \textbf{Keywords}: latex. abntex. text editoration.
			 \end{otherlanguage*}
			\end{resumo}
		}
		



		\label{oculto:listas}
		% ---
		% inserir lista de ilustrações
		% ---
		\ImprimirSimOuNao[Não]{
			\pdfbookmark[0]{\listfigurename}{lof}
			\listoffigures*
			\cleardoublepage
		}
		% ---





		% ---
		% inserir lista de tabelas
		% ---
		\ImprimirSimOuNao[Não]{
			%\pdfbookmark[0]{\listtablename}{lot}
			%\listoftables*
			%\cleardoublepage
		}
		% ---





		% ---
		% inserir lista de abreviaturas e siglas
		% ---
		\ImprimirSimOuNao[Não]{
			%\begin{siglas}
			%  \item[ABNT] Associação Brasileira de Normas Técnicas
			%  \item[abnTeX] ABsurdas Normas para TeX
			%\end{siglas}
		}
		% ---





		% ---
		% inserir lista de símbolos
		% ---
		\ImprimirSimOuNao[Não]{
			%\begin{simbolos}
			%  \item[$ \Gamma $] Letra grega Gama
			%  \item[$ \Lambda $] Lambda
			%  \item[$ \zeta $] Letra grega minúscula zeta
			%  \item[$ \in $] Pertence
			%\end{simbolos}
		}
		% ---



		\label{oculto:sumario-índice}
		% ---------------------
		% inserir o sumario    
		% ---------------------
		\ImprimirSimOuNao[Sim]{
			\pdfbookmark[0]{\contentsname}{toc}

			% Formatação forçada do sumário identado e dot juntinho
			\makeatletter 
				\renewcommand{\cftdotsep}{.5}
				\renewcommand{\cftlastnumwidth}{1em}
				\cftsetindents{chapter}{1em}{1.5em}
				\cftsetindents{section}{1.5em}{2em}
				\cftsetindents{subsection}{3.75em}{2.5em}
				\cftsetindents{subsubsection}{6.2em}{3.5em}
			\makeatother

			\tableofcontents*

			\cleardoublepage
		}
		
		




















		
% -----------------------------------------------------------------
% ELEMENTOS TEXTUAIS
% -----------------------------------------------------------------
\textual
		

\chapter{Introdução}
	\section{Objetivos}
		Efetuar uma reflexão a partir da leitura dos textos de \citeaa{Roda2018} e \citeaa{SantosWagner2013} referente ao tema do ensino e aprendizagem da análise combinatória, como elemento de compreensão e evolução da álgebra ensinada no ensino fundamental e médio.
		
		
	\section{Leitura} 
		Os dois textos tratam em maior ou menor grau da análise combinatória e sua propagação na sala de aula como complementação intelectual do professor.
		
		\citeaa{Roda2018} é uma tese de mestrado profissional em matemática que resumidamente, \textit{"Diante das dificuldades encontradas no ensino aprendizagem da Análise Combinatória nas escolas públicas, tanto para o aluno quanto para o professor do Ensino Médio, o presente trabalho ganhou forma para propor uma abordagem diferenciada. O trabalho tem como objetivo abordar a Análise Combinatória sem o uso das fórmulas, pois se tem percebido que Análise Combinatória é tratada por muitos como um ramo da Matemática que precisa decorar fórmulas, mecanizando o aprendizado dos alunos. Com isso, o conteúdo é intitulado como um dos mais difíceis em Matemática, portanto propomos trabalhar apenas com o Princípio Fundamental da Contagem (PFC) como ferramenta das resoluções dos exercícios. Trabalhamos a proposta na prática com uma sala de 2ª série do Ensino Médio, e obtemos resultados muito bons. Esperamos que o tranalho seja um material para estudos de professores e alunos."}
		
		Já \citeaa{SantosWagner2013} discute \textit{"as resoluções de 198 estudantes de quatro universidades baianas relacionadas à análise combinatória. Objetivamos no estudo identificar o que eles compreendiam de conceitos de combinatória, se sabiam diferenciar arranjo e combinação, que estratégias de resoluções empregaram e erros cometeram. Utilizamos trabalhos de análise de erros e taxionomia dos objetivos educacionais como aportes teóricos da pesquisa. Desenvolvemos uma pesquisa qualitativa e a análise de dados indicou que os estudantes universitários de 3º e 8º semestres ainda apresentavam dificuldades conceituais e procedimentais com respeito à combinatória."}


			
\chapter{Desenvolvimento}

	Fica latente nas leituras do texto que o desenvolvimento de linguagem, tanto a matemática quanto a língua natural, é imprescindível para o desenvolvimento matemático do aluno. Na minha experiência como educador sempre enfatizei que Português e Literatura não podem ser separados da matemática, tanto como veículo de transmissão do conhecimento como a formação das estruturas mentais necessárias para a abstração que a álgebra exige, incluindo o ensino de análise combinatória. 
	
	Da mesma forma, o uso filosófico da capacidade do aluno generalizar a partir do abstrato se mostra fonte rica de experimentação, tornando menos traumático a passagem da aritmética para a álgebra.
	
	Entendo que o português, associado ao ensino de matemática, notadamente nos problemas envolvendo contagem e análise combinatória, se mostra uma ferramenta fundamental para desistigmatizar a matemática como disciplina fria e sem vida. Já que da interpretação emerge a solução baseada nos princípios da contagem, como bem explica \citeaa{SantosWagner2013} \textit{... o desenvolvimento de uma nova forma de pensar em matemática denominada raciocínio combinatório. Ou seja, decidir sobre a forma mais adequada de organizar números ou informações para poder contar os casos possíveis não deve ser aprendido como uma lista de fórmulas, mas como um processo que exige a construção de um modelo simplificado e explicativo da situação.}
	
	
	
		

\chapter{Conclusão}

	Fica claro, na leitura dos textos, que existe um distanciamento ente a teoria pregada na formação de professores e a realidade de sala de aula. Notadamente, os professores dispõem de pouco tempo e currículos e materiais didáticos não se inter-relacionam de maneira satisfatória.
	
	Isso, na opinião desse, incorre nas chamadas "bolhas" de matéria escolar. Dentro do próprio currículo existe a chance da matemática tratar as matérias que usam a matemática como fonte mas isso, em geral, é artificialmente injetado na matéria de matemática. As vezes fica parecendo heresia trazer um conteúdo de química para uma aula de matemática. "Nossa... que absurdo, agora terei que aprender química para ensinar essa questão...", imagino um professor.
	
	Não faltam tentativas de unir as matérias afins, no youtube encontrei, com certa dificuldade, o vídeo \textbf{Ferramentas Matemáticas na Química \url{https://www.youtube.com/watch?v=qqreG99-iJQ} \citeaa{Stoodi2019}} que aborda o tema, logicamente com foco em concursos e ENEM, mas não deixa de ser um belo ponto de partida.

	Minha conclusão é que precisamos quebrar algumas bolhas, ou espaços compartimentados, para que a tão sonhada multidisciplinaridade, proposta pelos PNC e BNCC. Precisamos cada vez mais professores invadindo a área de outros professores. Meu sonho é trazer professores de português para falar de linguística associado a matemática, redação com foco em álgebra, aulas de química ministrado por professores de matemática para reforçar a ligação entre as matérias.
	
	Eu, como acadêmico, desde 1991 procuro uma forma de resgatar o treino na álgebra conhecido dos alunos do início do século. Não existia tanto o lúdico, mas os alunos eram expostos a mais exercícios. Com a matemática moderna, entra o lúdico e sai os exercícios. Precisamos encontrar um meio termo em que o aluno seja exposto mais ao prático, aos exercícios, com lúdico e com pensamento mais científico, onde estar menos errado é preponderante à dicotomia de estar sempre certo.
	
	Alias, esse ponto de tirar a matriz filosófica de estar certo sempre e, ao invés disso propor uma eduação em que estar menos errado, abraçando o erro e a mudança como pontos centrais. Tomo como base o vídeo \textbf{Por que é melhor estar menos errado e não mais certo? \url{https://www.youtube.com/watch?v=EoPdXo3hYHc} \citeaa{Souza2021}} e derivo que o próprio sistema educacional e moral nos obriga a "estarmos sempre certos" ao invés de "estarmos menos errado" e pode ser a matéria prima da atual polarização que vivemos hoje.
	
	
	Na minha visão devemos misturar mais o acadêmico formal na prática de ensino fundamental, logicamente com critério e direção formativa. Lembro-me do dia que descobri os logaritmos e compreendi que podia reduzir operações matemáticas a um nível abaixo, multiplicações em somas, potências em multiplicações mais simples. Noções conseguidas por alguém que espontaneamente pela pobreza ia mais a biblioteca e era mais chegado a livros por não ter acesso a diversões modernas como o computador.
	
	Concluo citanto \cite{SantosWagner2013} \textit{Muitos estudantes também relataram não haver estudado análise combinatória no ensino médio ou, quando o ensino ocorreu, informaram que esse foi realizado de forma mecânica, apenas recorrendo ao uso de fórmulas ou técnicas. Isso não garante a apreensão ou construção dos conceitos envolvidos em análise combinatória com compreensão e significado... É necessário que este conteúdo se faça presente na estrutura curricular dos cursos. Por outro lado, os estudantes também precisam modificar seus hábitos de estudo e recuperar o que não foi aprendido seja na educação básica ou no ensino superior. Faz-se necessária a dedicação de mais tempo no curso de licenciatura para que assim o interesse pelo estudo de combinatória seja despertado e conscientemente aprendido.}
	





\label{oculto:bibliografia}
%%---------------------------------------------------------------------
%% BIBLIOGRAFIA - Referências Bibliográficas
%%---------------------------------------------------------------------
% Referências bibliográficas
\bibliography{bibliografia/!!!bibliografia}	        % Elemento Obrigatório








\postextual

	% ----------------------------------------------------------
	% Glossário
	% ----------------------------------------------------------

	% Consulte o manual da classe abntex2 para orientações sobre o glossário.

	\ImprimirSimOuNao[Não]{             % Elemento Opcional
		\glossary
	}



	% ----------------------------------------------------------
	% Apêndices
	% ----------------------------------------------------------

	\ImprimirSimOuNao[Não]{             % Elemento Opcional
		% ---
		% Inicia os apêndices
		% ---
		\begin{apendicesenv}

		% Imprime uma página indicando o início dos apêndices
		\partapendices

		% ----------------------------------------------------------
		\chapter{Sobre esse trabalho}
		% ----------------------------------------------------------


			Esse documento foi produzido e programado usando-se \LaTeX, MikTeX, abntex2 e todo conteúdo possui links referencias clicáveis, sejam tabelas, figuras, imagens de vídeos, autores com seu respectivo registro bibliográfico.
			Informamos que o código fonte do projeto gerador desse PDF está disponível no endereço abaixo (legível também pelo QR Code abaixo): \\
			
			\qrset{link, height=4cm}
			\begin{center}
				\href{https://github.com/ChicoFigueiredo/Estacio-TCC-Estacio-Matematica}{
					\qrcode{https://github.com/ChicoFigueiredo/Estacio-TCC-Estacio-Matematica}
				}
			\end{center}
			\begin{center}
				{\tiny \url{https://github.com/ChicoFigueiredo/Estacio-TCC-Estacio-Matematica} }
			\end{center}
			


		% ----------------------------------------------------------
		%\chapter{Nullam elementum urna vel imperdiet sodales elit ipsum pharetra ligula
		%ac pretium ante justo a nulla curabitur tristique arcu eu metus}
		%% ----------------------------------------------------------
		%\lipsum[55-57]

		\end{apendicesenv}
		% ---
	}

	\label{oculto:anexos}
	% ----------------------------------------------------------
	% Anexos
	% ----------------------------------------------------------
	%
	% ---
	% Inicia os anexos
	% ---
	\ImprimirSimOuNao[Não]{             % Elemento Opcional
		\begin{anexosenv}

			% Imprime uma página indicando o início dos anexos
			\partanexos
			
			\makeatletter 
				\def\thesection{\alph{section}}
				\setcounter{chapter}{1}
			\makeatother
			
			\section{Conteúdo desse trabalho}


				Esse documento foi programado em \LaTeX, MikTeX, abntex2 e todo conteúdo possui links referenciais clicáveis, sejam tabelas, figuras, imagens de vídeos, autores com seu respectivo registro bibliográfico.
				Informamos que projeto gerador desse PDF está disponível no endereço (legível também pelo QR Code abaixo): \\
				\url{https://github.com/ChicoFigueiredo/estacio-Trab001-AASE-202004137859.git} \\
				\qrset{link, height=4cm}
				\begin{center}
					\href{https://github.com/ChicoFigueiredo/estacio-Trab001-AASE-202004137859.git}{
						\qrcode{https://github.com/ChicoFigueiredo/estacio-Trab001-AASE-202004137859.git}
					}
				\end{center}


				Apresentação no OneDrive: \url{https://1drv.ms/p/s!AgRBucATAhUblzAldnG4LGnWNV-r?e=ykIvGi} \\
				\begin{center}
					\href{https://1drv.ms/p/s!AgRBucATAhUblzAldnG4LGnWNV-r?e=ykIvGi}{
						\qrcode{https://1drv.ms/p/s!AgRBucATAhUblzAldnG4LGnWNV-r?e=ykIvGi}
					}
				\end{center}

				Vídeo no YouTube: \url{https://youtu.be/szsZ_Uuk1zk} \\
				\begin{center}
					\href{https://youtu.be/szsZ_Uuk1zk}{
						\qrcode{https://youtu.be/szsZ_Uuk1zk}
					}
				\end{center}


				\section{Códigos e Programas utilizados}

				\subsection{Programas em Python}
					programas em python

				\subsection{Programas em R}
					programas em r

				\section{Lista de questões}
					lista de questões

			
		\end{anexosenv}
	}
	
	
	%%---------------------------------------------------------------------
	%% INDICE REMISSIVO
	%%---------------------------------------------------------------------
	\ImprimirSimOuNao[Sim]{             % Elemento Opcional
		\phantompart
		\printindex
	}
	%---------------------------------------------------------------------


\end{document}

% -----------------------------------------------------------------
% Fim do Documento
% -----------------------------------------------------------------
