% Comprensão
\pdfminorversion=5
\pdfcompresslevel=9
\pdfobjcompresslevel=2

\documentclass[
	12pt,				% tamanho da fonte
	openright,			% capítulos começam em pág ímpar (insere página vazia caso preciso)
	%twoside,			% para impressão em recto e verso. Oposto a oneside
	oneside,
	a4paper,			% tamanho do papel.
	chapter=TITLE,		% títulos de capítulos convertidos em letras maiúsculas
	section=TITLE,		% títulos de seções convertidos em letras maiúsculas
	sumario=abnt-6027-2012,
	english,			% idioma adicional para hifenização
	brazil				% o último idioma é o principal do documento
]{abntex2}

\label{oculto:packages}
% ----------------------------------------------------------
% Pacotes básicos
% ----------------------------------------------------------
\usepackage{import}                            % package import tem o comando import que faz a importação com "novo path"
\usepackage{amsmath}							% Pacote matemático
\usepackage{amssymb}							% Pacote matemático
\usepackage{amsfonts}							% Pacote matemático
\usepackage{lmodern}							% Usa a fonte Latin Modern
\usepackage[T1]{fontenc}						% Selecao de codigos de fonte.
\usepackage[utf8]{inputenc}						% Codificacao do documento (conversão automática dos acentos)
\usepackage{lastpage}							% Usado pela Ficha catalográfica
\usepackage{indentfirst}						% Indenta o primeiro parágrafo de cada seção.
\usepackage[dvipsnames,svgnames,table]{xcolor}			% Controle das cores
\usepackage{graphicx}							% Inclusão de gráficos
\usepackage{microtype} 							% para melhorias de justificação
\usepackage{lipsum}								% para geração de dummy text
\usepackage[brazilian,hyperpageref]{backref}	% Paginas com as citações na bibl
%\usepackage[alf]{abntex2cite}					% Citações padrão ABNT
\usepackage[num]{abntex2cite}					% Citações padrão ABNT numérica
\usepackage{adjustbox}							% Pacote de ajuste de boxes
\usepackage{subcaption}							% Inclusão de Subfiguras e sublegendas
\usepackage{enumitem}							% Personalização de listas
\usepackage{siunitx}							% Grandezas e unidades
\usepackage[section]{placeins}					% Manter as figuras delimitadas na respectiva seção com a opção [section]
\usepackage{multirow}							% Multi colunas nas tabelas
\usepackage{array,tabularx} 					% Pacotes de tabelas
\usepackage{booktabs}							% Pacote de tabela profissonal
\usepackage{rotating}							% Rotacionar figuras e tabelas
\usepackage{xfrac}								% Fazer frações n/d em linha
\usepackage{bm}									% Negrito em modo matemático
\usepackage{xstring}							% Manipulação de strings
\usepackage{pgfplots}							% Pacote de Gráficos
\usepackage{tikz}								% Pacote de Figuras
\usepackage[american, cuteinductors,smartlabels, fulldiode, siunitx, americanvoltages, oldvoltagedirection, smartlabels]{circuitikz}						% Pacote de circuitos elétricos
\usepackage{lipsum}
\usepackage{xargs}
\usepackage{dsfont}

%\usepackage[style=abnt]{biblatex}

\usepackage{hyperref}
% informações do PDF
\makeatletter
	\hypersetup{
		%pagebackref=true,
		pdftitle={\@title},
		pdfauthor={\@author},
		pdfsubject={\imprimirpreambulo},
		pdfcreator={LaTeX with abnTeX2},
		pdfkeywords={abnt}{latex}{abntex}{abntex2}{trabalho academico},
		colorlinks=true,       		% false: boxed links; true: colored links
		linkcolor=NavyBlue,          	% color of internal links
		citecolor=NavyBlue,        	% color of links to bibliography
		filecolor=black,      		% color of file links
		urlcolor=NavyBlue,
		bookmarksdepth=4,
		linktoc=all
	}
\makeatother


% Centraliza captions of pictures
\usepackage[justification=centering]{caption}


% glossário, com fix para erros
%\usepackage[acronym]{glossaries}


% To use externalize consider
%https://tex.stackexchange.com/questions/182783/tikzexternalize-not-compatible-with-miktex-2-9-abntex2-package
%Lauro Cesar digged into the problem until he came with a solution for me to test. And it Works!
%
%According to this link:
%
%The package calc changed the commands \setcounter and friends to be fragile. So you have to make them robust. The example below uses etoolbox with \robustify:
%
\usepackage{etoolbox}
\robustify\setcounter
\robustify\addtocounter
\robustify\setlength
\robustify\addtolength

\usepackage[]{tocloft}
%\setlength\cftsectionnumwidth{4em}
%\setlength{\cftchapterindent}{2em}
%\setlength{\cftsectionindent}{5em}
%\setlength{\cftsubsectionindent}{8em}

\usepackage{pstricks-add}
\pgfplotsset{compat=1.15}
\usepackage{mathrsfs}
\usetikzlibrary{arrows}

%<<<<<<<WARNING>>>>>>>
% PGF/Tikz doesn't support the following mathematical functions:
% cosh, acosh, sinh, asinh, tanh, atanh,
% x^r with r not integer

% Plotting will be done using GNUPLOT
% GNUPLOT must be installed and you must allow Latex to call external
% programs by adding the following option to your compiler
% shell-escape    OR    enable-write18 
% Example: pdflatex --shell-escape file.tex 


%% How to silence memoir class warning against the use of caption package?
%% https://tex.stackexchange.com/questions/391993/how-to-silence-memoir-class-warning-against-the-use-of-caption-package
%\usepackage{silence}
%\WarningFilter*{memoir}{You are using the caption package with the memoir class}
%\WarningFilter*{Class memoir Warning}{You are using the caption package with the memoir class}


% -----------------------------------------------------------------
% Você pode adicionar seus pacotes a partir desta linha;
% -----------------------------------------------------------------

%\usepackage[showframe,pass]{geometry}
%\usepackage[11,12]{pagesel}
\usepackage{qrcode}
\usepackage{multirow}

% -----------------------------------------------------------------
% pacote interno
\usepackage{modulos/trabalhos.academicos}



% CONFIGURAÇÕES DE PACOTES
% Configurações do pacote backref
% Usado sem a opção hyperpageref de backref
\renewcommand{\backrefpagesname}{Citado na(s) página(s):~}

% Texto padrão antes do número das páginas
\renewcommand{\backref}{}

% Define os textos da citação
\renewcommand*{\backrefalt}[4]{
	\ifcase #1 %
		Nenhuma citação no texto.%
	\or
		Citado na página #2.%
	\else
		Citado #1 vezes nas páginas #2.%
	\fi
}%

% Citação online --- MODIFICAR ---
\newcommandx{\citeaa}[2][2= ]{\citeauthoronline{#1}~(\citeyear{#1}#2)}


\label{oculto-programinhas}
%-----------------------------------------
% (1) simple command for print or not
%-----------------------------------------
\usepackage{ifthen}
\newcommand{\ImprimirSimOuNao}[2][Sim]
{
  \ifthenelse{\equal{#1}{Sim}}{#2}{}
}

% alterando o aspecto da cor azul
%\definecolor{blue}{RGB}{41,5,195}

\makeatletter
    \newcommand{\includetikz}[1]{%
    	\tikzsetnextfilename{#1}%
    	\input{#1.tex}%
    }
\makeatother


\newcommand{\MONTH}{%
	\ifcase\the\month
	\or JAN% 1
	\or FEV% 2
	\or MAR% 3
	\or ABR% 4
	\or MAI% 5
	\or JUN% 6
	\or JUL% 7
	\or AUG% 8
	\or SET% 9
	\or OUT% 10
	\or NOV% 11
	\or DEZ% 12
	\fi}
\makeatletter

% Personalização das opções das listas
\setlist[itemize]{leftmargin=\parindent}

\newcommand{\me}[1]{Elaborado pelo autor, #1.}

% Configuração do pgfplots
\pgfplotsset{compat=newest} %compat=1.14
\pgfplotsset{plot coordinates/math parser=false}
\newlength\figureheight
\newlength\figurewidth

% Libraries do TiKz
\usetikzlibrary{quotes,angles,arrows}
\usetikzlibrary{through,calc,math}
\usetikzlibrary{graphs,backgrounds,fit}
\usetikzlibrary{shapes,positioning,patterns,shadows}
\usetikzlibrary{decorations.pathreplacing}
\usetikzlibrary{shapes.geometric}
\usetikzlibrary{arrows.meta}
\usetikzlibrary{external}

%\tikzexternalize[]
%\tikzexternalenable
%\tikzexternalize
%\tikzexternaldisable
%\tikzset{external/force remake}
%\tikzexternalize[shell escape=-enable-write18]

% Configurações do CircuiTiKz
\ctikzset{bipoles/thickness=1}
%\ctikzset{bipoles/length=1.2cm}
\ctikzset{monopoles/ground/width/.initial=.2}
\ctikzset{bipoles/resistor/height=0.25}
\ctikzset{bipoles/resistor/width=0.6}
\ctikzset{bipoles/capacitor/height=0.5}
\ctikzset{bipoles/capacitor/width=0.15}
\ctikzset{bipoles/generic/height=0.25}
\ctikzset{bipoles/generic/width=0.6}
%\ctikzset{bipoles/capacitor polar/length=0.5}
%\ctikzset{bipoles/diode/height=.375}
%\ctikzset{bipoles/diode/width=.3}
%\ctikzset{tripoles/thyristor/height=.8}
%\ctikzset{tripoles/thyristor/width=1}
\ctikzset{bipoles/vsourcesin/height=.5}
\ctikzset{bipoles/vsourcesin/width=.5}
\ctikzset{bipoles/cvsourceam/height=.6}
\ctikzset{bipoles/cvsourceam/width=.6}
%\ctikzset{tripoles/european controlled voltage source/width=.4}

\tikzstyle{every node}=[font=\footnotesize]
\tikzstyle{every path}=[line width=0.25pt,line cap=round,line join=round]
%\tikzstyle{every path}=[line cap=round,line join=round]


% Definição de cores MATLAB
\definecolor{matlab_blue}{rgb}	{         0,    0.4470,    0.7410}
\definecolor{matlab_orange}{rgb}{    0.8500,    0.3250,    0.0980}
\definecolor{matlab_yellow}{rgb}{    0.9290,    0.6940,    0.1250}
\definecolor{matlab_violet}{rgb}{    0.4940,    0.1840,    0.5560}
\definecolor{matlab_green}{rgb}	{	 0.4660,    0.6740,    0.1880}
\definecolor{matlab_lblue}{rgb}	{    0.3010,    0.7450,    0.9330}
\definecolor{matlab_red}{rgb}	{    0.6350,    0.0780,    0.1840}

% Personalização das legendas
\usepackage[format = hang,
			labelsep = endash,
			singlelinecheck = false,
			skip = 6pt,
			listformat = simple]{caption}

% Personalização das unidades
\sisetup{output-decimal-marker = {,}}
\sisetup{exponent-product = \cdot, output-product = \cdot}
\sisetup{tight-spacing=true}
\sisetup{group-digits = false}

% Personalizações de tipo de colunas de tabelas
\newcolumntype{L}[1]{>{\raggedright\let\newline\\\arraybackslash\hspace{0pt}}m{#1}}
\newcolumntype{C}[1]{>{\centering\let\newline\\\arraybackslash\hspace{0pt}}m{#1}}
\newcolumntype{R}[1]{>{\raggedleft\let\newline\\\arraybackslash\hspace{0pt}}m{#1}}

% Personalizações de cores da UDESC
\definecolor{CapaAmareloUDESC}{RGB}{243,186,83}		% Especializacao
\definecolor{CapaVerdeUDESC}{RGB}{0,112,52}			% Mestrado
\definecolor{CapaVermelhoUDESC}{RGB}{171,35,21}		% Doutorado
\definecolor{CapaAzulUDESC}{RGB}{38,54,118} 		% Pós-Doutorado


% Espaçamento depois do título
\setlength{\afterchapskip}{0.7\baselineskip}
% O tamanho do parágrafo é dado por:
\setlength{\parindent}{1.25cm}
% Controle do espaçamento entre um parágrafo e outro:
\setlength{\parskip}{0.15cm}  % tente também \onelineskip
%\SingleSpacing % Espaçamento simples
\OnehalfSpacing % Espaçamento 1,5 (UDESC/CCT)
%\DoubleSpacing	% Espaçamento duplo

% ---
% Margens - NBR 14724/2011 - 5.1 Formato
% ---
\setlrmarginsandblock{3cm}{2cm}{*}
\setulmarginsandblock{3cm}{2cm}{*}
\checkandfixthelayout[fixed]
% ---









\label{oculto:dados-do-trabalho}
% -----------------------------------------------------------------
% Informações de dados para CAPA e FOLHA DE ROSTO
% -----------------------------------------------------------------
\tipotrabalho{
    Atividade Estruturada
}

\titulo{
	Resenha da importância do conteúdo de Calculo I na formação de professores
}%

% ATENÇÃO: O símbolo {} indica o sobrenome para a ficha catalográfica.
%          Exemplo: Sherlock Holmes {}da Silva para sobrenomes compostos;
%          Exemplo: Arnold Alois {}Schwarzenegger para sobrenome simples.
\autor{Francisco Lima {}Figueiredo}%
\newcommand\matricula{Matricula 202201097825}
\orientador{Andre Luis {}Funcke}%
%\coorientador{}%
\instituicao{Universidade Estácio de Sá}%

\preambulo{
	Trabalho apresentado ao professor Andre Luis Funcke como requisito parcial para obtenção da aprovação na disciplina 
    Cálculo I (CEL1397) 9001.
}

\local{Brasília}%
\data{\MONTH/\the\year}%
% ---










% compila o indice
\makeindex

% -----------------------------------------------------------------
% Início do documento
% -----------------------------------------------------------------
\begin{document}
	
	\selectlanguage{brazil}
	\frenchspacing            % Retira espaço extra obsoleto entre as frases.
	
	% Você pode comentar os elementos que não deseja em seu trabalho;

	% -----------------------------------------------------------------
	% ELEMENTOS PRÉ-TEXTUAIS
	% -----------------------------------------------------------------
	\pretextual

	\label{oculto:capa}
	% --- 
	% Capa
	% --- 
	%\imprimircapaTrabalho      % Capa UDESC para Trabalho
	%\imprimircapaTCC			% Capa UDESC para TCC
	%\imprimircapaDissertacao	% Capa UDESC para Dissertações
	%\imprimircapaTese			% Capa UDESC para Teses
	%\imprimircapaPosDoc		% Capa UDESC para Pós-Doutorado

	\imprimirCapaCustom{CapaAmareloUDESC}{\imprimirtipotrabalho}{\imprimirinstituicao}  %Capa UDESC para Pós-Doutorado

	%\imprimircapa				% Capa padrão


	% -----------------------------------------------------------------
	% ELEMENTOS PRÉ-TEXTUAIS
	% -----------------------------------------------------------------

		% ---
		% Folha de rosto
		% (o * indica que haverá a ficha bibliográfica)
		% ---
		\ImprimirSimOuNao[Sim]{             % Elemento Obrigatório
			\imprimirfolhaderosto*                                                 
		}
		% ---




		\label{oculto:ficha-bibliográfica}
		% ---
		% Inserir a ficha bibliografica
		% ---
		\ImprimirSimOuNao[Não]{             % Elemento Obrigatório

			% Isto é um exemplo de Ficha Catalográfica, ou ``Dados internacionais de
			% catalogação-na-publicação''. Você pode utilizar este modelo como referência.
			% Porém, provavelmente a biblioteca da sua universidade lhe fornecerá um PDF
			% com a ficha catalográfica definitiva após a defesa do trabalho. Quando estiver
			% com o documento, salve-o como PDF no diretório do seu projeto e substitua todo
			% o conteúdo de implementação deste arquivo pelo comando abaixo:

			% \begin{fichacatalografica}
			%     \includepdf{fig_ficha_catalografica.pdf}
			% \end{fichacatalografica}

			%	\setlength{\parindent}{0cm}
			%	\setlength{\parskip}{0pt}
			\begin{fichacatalografica}                            
				%\sffamily
				%\rmfamily
				\ttfamily \hbadness=10000
				\vspace*{\fill}					% Posição vertical
				\begin{center}					% Minipage Centralizado
			%	\begin{minipage}[c]{8cm}
			%	\centering \sffamily
			%	 Ficha catalográfica elaborada pelo(a) autor(a), com auxílio do programa de geração automática da Biblioteca Setorial do CCT/UDESC
			%	\end{minipage}
				\fbox{\begin{minipage}[c]{12.5cm}		% Largura
				\flushright
				{\begin{minipage}[c]{10.5cm}		% Largura
				\vspace{1.25cm}
				%\footnotesize
				\setlength{\parindent}{1.5em}
				\noindent \invertname{\imprimirautor} \par
				\imprimirtitulo{ }/{ }\imprimirautor. -- \imprimirlocal, \imprimirdata .\par
				\pageref{LastPage} p. : il. ; 30 cm.\par
				\vspace{1.5em}
				\imprimirorientadorRotulo~\imprimirorientador\par
				\imprimircoorientadorRotulo~\imprimircoorientador\par
				\imprimirtipotrabalho~--~\imprimirinstituicao, \imprimirlocal, \imprimirdata.\par
				\vspace{1.5em}
					1. Sociologia.
					2. Meio Ambiente.
					I. \invertname{\imprimirorientador}.
					II. \invertname{\imprimircoorientador}.
					III. \imprimirinstituicao. %
				\vspace{1.25cm}	%
				\end{minipage}%
				}%
				\end{minipage}}%
				\end{center}
			\end{fichacatalografica}
		}




		% ---
		% Inserir errata
		% ---
		\ImprimirSimOuNao[Não]{
			\begin{errata}
				Elemento opcional da \citeonline[4.2.1.2]{NBR14724:2011}. Exemplo:

				\vspace{\onelineskip}

				FERRIGNO, C. R. A. \textbf{Tratamento de neoplasias ósseas apendiculares com
				reimplantação de enxerto ósseo autólogo autoclavado associado ao plasma
				rico em plaquetas}: estudo crítico na cirurgia de preservação de membro em
				cães. 2011. 128 f. Tese (Livre-Docência) - Faculdade de Medicina Veterinária e
				Zootecnia, Universidade de São Paulo, São Paulo, 2011.

				\begin{table}[htb]
				\center
				\footnotesize
				\begin{tabular}{|p{1.4cm}|p{1cm}|p{3cm}|p{3cm}|}
				  \hline
				   \textbf{Folha} & \textbf{Linha}  & \textbf{Onde se lê}  & \textbf{Leia-se}  \\
					\hline
					1 & 10 & auto-conclavo & autoconclavo\\
				   \hline
				\end{tabular}
				\end{table}
				
			\end{errata}
			% ---
		}




		
		% ---
		% Inserir folha de aprovação
		% ---
		\ImprimirSimOuNao[Não]{
			\begin{folhadeaprovacao}
				\begin{center}
					{\MakeTextUppercase{\ABNTEXchapterfont\large\imprimirautor}}

					\vspace*{\fill} %\vspace*{\fill}
					\begin{center}
					  {\MakeTextUppercase{\ABNTEXchapterfont\bfseries\large\imprimirtitulo}}
					\end{center}
					\vspace*{\fill}
					
					%\hspace{.45\textwidth}
					{\begin{minipage}[c]{1\linewidth}
						\setlength{\parindent}{1.25cm}
						\imprimirpreambulo
					\end{minipage}}%
					\vspace*{\fill}
					\end{center}
						
					 
					{\bfseries Banca Examinadora: }
					\vspace*{\fill}
					
					{Orientador: \vspace{-16pt} }
					\assinatura{\textbf{Prof. \imprimirorientador , Dr.} \\ Univ. XXX} 
					{Coorientador: \vspace{-16pt}}   
					\assinatura{\textbf{Prof. \imprimircoorientador , Dr.} \\ Univ. XXX}
				   
					{Membros: \vspace{-16pt} } 
					
				% --- Exemplo de assinaturas em sequência ---       
					\setlength{\ABNTEXsignwidth}{8.5cm}
					
					\assinatura{\textbf{Prof. Professor, Dr.} \\ Univ. XXX}
					\assinatura{\textbf{Prof. Professor, Dr.} \\ Univ. XXX}
					\assinatura{\textbf{Prof. Professor, Dr.} \\ Univ. XXX}

				% --- Exemplo de assinaturas lado a lado ---
					\setlength{\ABNTEXsignwidth}{7.5cm}
				%
				%    \noindent\hfill\assinatura*{\textbf{Prof. Professor, Dr.} \\ Univ. XXX}%
				%    \hfill%
				%    \assinatura*{\textbf{Prof. Professor, Dr.} \\ Univ. XXX}%
				%    \hfill
				%    
				%    \noindent\hfill\assinatura*{\textbf{Prof. Professor, Dr.} \\ Univ. XXX}%
				%    \hfill%
				%    \assinatura*{\textbf{Prof. Professor, Dr.} \\ Univ. XXX}%
				%    \hfill
					
					\vspace*{\fill}  
					\begin{center}
					{\large\imprimirlocal, 01 de maio de \imprimirdata}
				\end{center}
				\vspace*{1cm}  
			\end{folhadeaprovacao}
		}
		% ---




		\label{oculto:dedicatória}
		% ---
		% Dedicatória
		% ---
		\ImprimirSimOuNao[Não]{
			\begin{dedicatoria}
			   \vspace*{\fill}
			   \centering
			   \noindent
			   \textit{ Este trabalho é dedicado às minhas maiores inspirações,\\
			   Minha esposa Daiane, minha filha Mayara Barros e minha sobrinha do coração Rachael Costa.} \vspace*{\fill}
			\end{dedicatoria}		
		}
		% ---
		
		
		
		
		\label{oculto:agradecimentos}
		% ---
		% Agradecimentos
		% ---
		\ImprimirSimOuNao[Não]{
			\begin{agradecimentos}
				Os agradecimentos principais são direcionados à Gerald Weber, Miguel Frasson, Leslie H. Watter, Bruno Parente Lima, Flávio de Vasconcellos Corrêa, Otavio Real Salvador, Renato Machnievscz\footnote{Os nomes dos integrantes do primeiro projeto abn\TeX\ foram extraídos de \url{http://codigolivre.org.br/projects/abntex/}} e todos aqueles que contribuíram para que a produção de trabalhos acadêmicos conforme as normas ABNT com \LaTeX\ fosse possível.

				Agradecimentos especiais são direcionados ao Centro de Pesquisa em Arquitetura da Informação\footnote{\url{http://www.cpai.unb.br/}} da Universidade de Brasília (CPAI), ao grupo de usuários \emph{latex-br}\footnote{\url{http://groups.google.com/group/latex-br}} e aos novos voluntários do grupo \emph{\abnTeX}\footnote{\url{http://groups.google.com/group/abntex2} e \url{http://www.abntex.net.br/}}~que contribuíram e que ainda contribuirão para a evolução do \abnTeX.
			\end{agradecimentos}
		}
		% ---




		\label{oculto:epigrafe}
		% ---
		% Epígrafe
		% ---
		\ImprimirSimOuNao[Sim]{
			\begin{epigrafe}
				\vspace*{\fill}
				\hspace{.35\textwidth}
				{\begin{minipage}{.6\textwidth}
					\begin{flushright}
						\textit{``A fortuna troca, às vezes, os cálculos da natureza.``\\
							(Machado de Assis)}
					\end{flushright}
				\end{minipage}}%
			\end{epigrafe}
		}
		% ---
		
		
		
		
		\label{oculto:resumo}
		% ---
		% RESUMOS
		% ---
		\ImprimirSimOuNao[Não]{
			% resumo em português
			\setlength{\absparsep}{18pt} % ajusta o espaçamento dos parágrafos do resumo
			\begin{resumo}
			 O presente estudo tem como objetivo coletar diversas estatísticas, parâmetros e métricas de provas do ENEM de vários anos e discorrer sobre a prova, o desempenho na disciplina de matemática e os eventuais fatores e eventos correlatos que possam indicar causa-efeito que afetem a nota final da disciplina na avaliação nacional. \\
			 Foram analisadas provas entre os anos de 2015 a 2019 do ENEM, seus gabaritos e os microdados de cada aluno, o que fornece um ambiente rico para perguntas e testes de hipóteses.\\
			 

			 \textbf{Palavras-chave}: ENEM, estatísticas, habilidades, competências, métricas, parâmetros.
			\end{resumo}
		}
		
		
		
		
		\label{oculto:abstract}
		% ---
		% Abstract
		% ---
		\ImprimirSimOuNao[Não]{
			% resumo em inglês
			\begin{resumo}[Abstract]
			 \begin{otherlanguage*}{english}
			   This is the english abstract.

			   \vspace{\onelineskip}
			 
			   \noindent 
			   \textbf{Keywords}: latex. abntex. text editoration.
			 \end{otherlanguage*}
			\end{resumo}
		}
		



		\label{oculto:listas}
		% ---
		% inserir lista de ilustrações
		% ---
		\ImprimirSimOuNao[Não]{
			\pdfbookmark[0]{\listfigurename}{lof}
			\listoffigures*
			\cleardoublepage
		}
		% ---





		% ---
		% inserir lista de tabelas
		% ---
		\ImprimirSimOuNao[Não]{
			%\pdfbookmark[0]{\listtablename}{lot}
			%\listoftables*
			%\cleardoublepage
		}
		% ---





		% ---
		% inserir lista de abreviaturas e siglas
		% ---
		\ImprimirSimOuNao[Não]{
			%\begin{siglas}
			%  \item[ABNT] Associação Brasileira de Normas Técnicas
			%  \item[abnTeX] ABsurdas Normas para TeX
			%\end{siglas}
		}
		% ---





		% ---
		% inserir lista de símbolos
		% ---
		\ImprimirSimOuNao[Não]{
			%\begin{simbolos}
			%  \item[$ \Gamma $] Letra grega Gama
			%  \item[$ \Lambda $] Lambda
			%  \item[$ \zeta $] Letra grega minúscula zeta
			%  \item[$ \in $] Pertence
			%\end{simbolos}
		}
		% ---



		\label{oculto:sumario-índice}
		% ---------------------
		% inserir o sumario    
		% ---------------------
		\ImprimirSimOuNao[Sim]{
			\pdfbookmark[0]{\contentsname}{toc}

			% Formatação forçada do sumário identado e dot juntinho
			\makeatletter 
				\renewcommand{\cftdotsep}{.5}
				\renewcommand{\cftlastnumwidth}{1em}
				\cftsetindents{chapter}{1em}{1.5em}
				\cftsetindents{section}{1.5em}{2em}
				\cftsetindents{subsection}{3.75em}{2.5em}
				\cftsetindents{subsubsection}{6.2em}{3.5em}
			\makeatother

			\tableofcontents*

			\cleardoublepage
		}
		
		




















		
% -----------------------------------------------------------------
% ELEMENTOS TEXTUAIS
% -----------------------------------------------------------------
\textual
		

\chapter{Introdução}
	\section{Objetivos}
		Efetuar uma reflexão a partir da leitura dos textos de \citeaa{Silva2011} e \citeaa{Goncalves2013} referente ao tema da importância dos conteúdos de cálculo diferencial e integral na formação de professores de matemática. 
		
		
	\section{Leitura} 
		\citeaa{Goncalves2013} apresenta em seu artigo intitulado \textit{"Atividades Investigativas de Aplicações das
			Derivadas Utilizando o GeoGebra"} um produto educacional e uma discussão sobre atividades exploratórias do uso do software GeoGebra para assuntos ensinados na gradução sobre tópicos de derivadas e integrais.
		
		Após relatar várias experiências e pontos de vista sobre o ensino de derivada e seu descolamento das aplicações práticas dos conceitos, se atendo ao formalismo exagerado que o tema exige sobre os assuntos de função e limite. E apesar de mencionar que softwares como o Geogebra tem 'potencial pedagógico' por meio de exploração e manipulação dinâmica de tópicos, não ficando presos a figuras estáticas e a mercê da imaginação do aluno (imaginem o mesmo que eu).
		
		O fato de haver esses softwares, não devemos abandonar outras formas de estimular a procura dos conceitos pelos alunos. Importante ressaltar o tempo ganho com cálculos e desenhos e ter esse tempo adicional para estudar conceitos e fixar conceitos.
		
		Apresentam também o produto educacional por meio de Dissertação do Mestrado Profissional em Educação Matemática, do programa
		de pós-graduação da Universidade Federal de Ouro Preto (UFOP), intitulada \textit{Aplicações das Derivadas no Cálculo I: Atividades investigativas utilizando o GeoGebra}. Esse produto tem várias contribuições, como ressignificação dos conhecimentos de cálculo, criação de ambiente de aprendizagem diferenciado e o encaminhamento da formação de um novo professor de matemática para os ensinos fundamental e médio.
		
		Já \citeaa{Silva2011} em seu artigo intitulado \textit{'Diferentes dimensões do ensino e aprendizagem do Cálculo'} aborda os diferentes componentes envolvidos no ensino e na aprendizagem do Cálculo e suas consequências para o próprio saber matemático.
		
		Após breve exposição sobre a história do cálculo e sua propensa relação com as dificuldades inerentes aos conceitos da matéria, como funções e infinitesimais, o autor apresenta 10 trabalhos de pesquisa acerca de conteúdos de limites, derivadas e integrais associados aos conhecimentos dos alunos de diversos níveis, desde completos desconhecedores do conteúdo a alunos que já estudaram a matéria, apresentando possíveis consequências dos conteúdos na compreensão mais ampla da matéria cálculo.  
		

			
\chapter{Desenvolvimento}

	É consenso nos 2 artigos a importância da disciplina de Cálculo para a formação dos alunos universitários, seja na cadeira de Matemática, seja nas carreiras afins como engenharia, química, etc... . O corpo de conhecimento de cálculo, bem como seus usos, é preocupação dos pesquisadores. Todos 'sabem' que o cálculo é uma ferramenta matemática de suma importância, mas a falta de conhecimento de suas aplicações, sejam práticas e teóricas, podem frustar os alunos ou deixa-los sem dar importância aos conceitos aprendidos, aprender apenas cálculo pelo cálculo.
	
	Seja por aplicações práticas, como as diretas do cálculo de velocidade de corpos em movimentos, sejam em equações diferenciais como interpretar a dinâmica de crescimento do volume de células de um tumor ao longo do tempo ($ \dfrac{dV}{dt} = \lambda V $).
	
	Como muitos tópicos de matemática, existe um efeito cascata de aprendizagem, tal como cálculo depender de limites, limites depender da álgebra fundamental nos anos de ensino fundamental de médio. Sem vivência e compreensão de situações problemas, cuja bagagem é formada de forma piramidal de conhecimento, com base sólida.
	
	Formar alunos que consigam navegar entre os diversos registros da derivada, ou seja, suas diversas representações algébricas, gráficos e geométrico em suas diversas aplicações, é imperativo para que o aluno possa compreender suas aplicações práticas em seu campo de atuação. Isso é especialmente grave se o professor de matemática, formado nos rigoroso formalismo matemático, ensina cálculo a alunos de outros campos do conhecimento e não interage de maneira prática o conteúdo com seu uso para formação daquele aluno.
	
	O uso de ferramentas de software construídos especialmente para o ensino de cálculo, por meio de construções dinâmicas que permitem a interação do aluno e, se bem conduzidos pelos mestres, permitir relacionar os elementos de causa e efeito das diversas interpretações de derivadas e suas aplicações.
	
	Importante que o educador deve ser tão explorador como seus alunos, já que o aprendizado deve ser encarado, e incentivado, como ato contínuo, estimulado e praticado pelos professores. Isto posto, o aproveitamento de material já pronto, como a criação de conteúdo pode ser um estímulo para encontrar formas adequadas de ensinar o conteúdo. Isso é especialmente desejado para que o professor possa prever e conduzir o aluno na sua exploração.
	
	Os pesquisadores mostram que essas ferramentas não devem ser as únicas ferramentas a serem utilizadas. É importante navegar em todas as formas de aprendizado para trazer os alunos tanto as ferramentas clássicas quanto as modernas. Alguns conceitos do Cálculo carregam em si uma 	dinamicidade que, muitas vezes, não é possível de ser observada utilizando-se apenas quadro e giz e/ou pincel. Utilizando-se de uma ferramenta computacional, os conceitos da disciplina podem ser explorados de forma a facilitar a compreensão dos mesmos.
	
	Com o uso da informática os alunos se preocupam menos com as operações e a parte técnica, possibilitando ao professor uma exploração diferenciada do conteúdo. 
	
	É salutar que, após a posse desse conhecimento, é de se esperar a formação de um "novo" professor de matemática, ou de outras matérias, baseadas na experimentação como formador pedagógico. \citeaa{Goncalves2013} é categórico ao afirmar que:
	
	
	\begin{citacao} 
		...nossa crença de que o desenvolvimento de atividades investigativas utilizando TICE’s pode contribuir para os processos de ensino e aprendizagem de Cálculo I que é fundamental na formação do professor de Matemática.
	\end{citacao}	
	
	Esses aspectos reforçam os aspectos multidisciplinar da matemática, em especial os tópicos de cálculo, vários estudos se preocupam com a formação de professores do ensino básico. Encorajar os alunos a descobrir os resultados pelas próprias vias, deixando os mecanicismos impostos pelos longos currículos e pouco tempo para ensinar.
	
	Interessante a constatação do aumento da quantidade de universitários em todas as áreas de conhecimento, sem o proporcional aumento no número de estudantes de matemática e, para esses estudantes, a diminuição das notas médias nos primeiros semestres. Obviamente a falta de investimento governamental é um ponto de atenção que impacta negativamente na formação e ensino.
	
	A tensão mencionada entre o "Matemático" e o "Educador Matemático" e seus confrontos entre esses saberes, entre o rígido formalismo e o didatismo complacente, se revela interessante ponto de pesquisa. O desprezo também de educadores por congressos e troca de conhecimento também é preocupante.
	
	As dificuldades de natureza histórico-epistemológica\footnote{referente à teoria do conhecimento, a relação entre o sujeito, ser pensante, e o objeto, ser inerte} já que se tem estudos que encaminhariam ao cálculo desde o séc. II a.C. nos seus cálculos de áreas e volumes, além da quadratura do círculo usando o método da exaustão.
	
	Salutar comentar os conceitos introduzidos por Leibniz, como função, diferencial (dx, dy, ...) e o desenvolvimento do Teorema Fundamental do Cálculo e o conceito de continuidade definida por Cauchy. 
	
	\citeaa{Silva2011} reforça que as dificuldades quanto à aprendizagem dos conteúdos envolvidos na disciplina Cálculo Diferencial e Integral, que compõe a grade curricular de cursos de Exatas em diferentes áreas, se traduz pelo alto índice de reprovação e desistência do curso inicialmente escolhido pelo jovem universitário.
	
	Do lado dos docentes, existem expectativas de desempenho dos alunos que em geral são frustadas. Existe um lag entre o que se ensina e que o aluno compreende. Essas expectativas podem ser frustadas pois não se conhece interações fora das escolas de aplicação entre professores dos ensinos universitários e do ensino básico. Fato que essa transição do ensino básico e universitário é um sério problema de pesquisa.
	
	A disputa aparente entre as rigorosas demonstrações e generalizações matemáticas batem de frente às práticas algorítmicas muitas vezes praticadas no ensino fundamental. Embora muitos professores de Cálculo também priorizem ou reforcem os procedimentos algorítmicos, a própria natureza da matéria manifesta seu caráter globalista e exija saberes isolados ou compartimentados do ensino médio.
	
	\citeaa{Silva2011} ao detalhar os componentes do projeto, segmenta em 10 subprojetos que trabalham os seguintes aspectos:
	
	\begin{enumerate}
		\item Os obstáculos epistemológicos à aprendizagem de limite. São previstos a abordagem do obstáculo geométrico, o estudo da influência do conceito de função, a exploração do horror ao infinito e de outros obstáculos inerentes ao conceito.
		
		\item A transição da educação básica para o ensino superior: Pretende-se investigar a coordenação de representação semiótica de conceitos e os conhecimentos mobilizados por alunos no curso de Cálculo.
		
		\item Os significados atribuídos à variável (incógnita, termo geral, variáveis relacionadas,
		etc.) por alunos da educação básica e da universidade e por professores dos diferentes
		níveis de ensino.
		
		\item A passagem do estudo de função de uma variável para o de mais de uma e seus efeitos
		no ensino e aprendizagem das derivadas parciais e da integral dupla.
		
		\item A interferência do papel da relação funcional das variáveis na compreensão da derivada.
		As representações de estudantes, professores e livros didáticos sobre integral.
		
		\item As representações de professores, estudantes e livros didáticos referentes ao ensino e
		aprendizagem do Teorema Fundamental do Cálculo. Explora-se o entendimento do
		papel das variáveis x e t na definição da função $\displaystyle F: [a, b] \rightarrow \mathds{R} $ , com $\displaystyle F(x) = \int\limits_a^x { f(t) dt} $, e 	possíveis consequências na relação recíproca entre a derivada e a integral, cerne do Teorema Fundamental do Cálculo.
		
		\item A evolução histórica do número $\pi$, visando ao estudo da gênese desse número relacionada com a questão da quadratura do círculo e a construção dos números reais.

		\item A modelagem, vista como estratégia de ensino, para o estudo de conceitos matemáticos.
		
		\item A questão da transição da educação básica para o ensino superior.
		
		\item A disciplina inicial de Cálculo em cursos de matemática em universidades brasileiras e seu papel na formação de professores, a partir de 1934.
		
	\end{enumerate}
	
	Dos resultados dos subprojetos acima, destaco como o entendimento dos alunos especialmente no tocante aos conceitos como 'limite', 'limitado', 'tende a'. Fora os alunos que tendem em desassociar conceitos matemáticos dos conceitos físicos como no subprojeto 1.
	
	Igualmente intrigante foi a imediatice de alunos associando a expressão algébrica solta $ x+7 $ imediatamente com a  $ x + 7 = 0 $ levando a $ x = -7 $, sem contudo entender que a expressão  $ x + 7 $ tem seu significado quando mencionado no subprojeto 2 . Interessante que após o uso e intervenção de softwares educativos os alunos tiveram outra perspectiva.
	
	No subprojeto seguinte, com relação a passagem do estudo de funções de uma para várias variáveis, revela que os alunos não tem intimidade com 3 eixos ou dimensões em gráficos, se perdem até que conseguem consolidar o conceito.
	
	Nos subprojetos 4 e 5 ficam claro a ruptura entre o mecanicismo mecânico dos algoritmos com novos conceitos exploratórios que derivada, integral e função tem. Tal mecanicismo é evidente quando da avaliação da expressão $\displaystyle \int\limits_{-2}^{2} \sqrt{4-x^2} dx $ era igual a $2\pi$ pelo simples fato de reconhecer que $\displaystyle \sqrt{4-x^2}$ ser na geometria analítica um semicírculo com centro na origem e raio 2, ou seja a área sob o semicírculo metade de $\displaystyle 2 \pi R \Rightarrow \frac{4\pi}{2} = 2\pi $.
	
	O subprojeto 6 tenta descobrir as relações das variáveis $x$ e $t$ na função $\displaystyle F: [a, b] \rightarrow \mathds{R} $ dada por $\displaystyle F(x) = \int\limits_a^x { f(t) dt} $, suas relações com o Teorema Fundamental do Cálculo que nem sempre é enfatizada. A pesquisa sobre as representações de professores procurou enfatizar o papel da identificação das variáveis visuais pertinentes, na conversão do registro gráfico para o algébrico e vice-versa e nas argumentações na língua natural.
	
	Sobre a gênese do número $ \pi $, do subprojeto 7, relacionada com a questão da quadratura do círculo e a construção dos números reais, tanto a pesquisa bibliográfica quanto a pesquisa relacionando a densidade dos números reais para investigar a reação dos professores do ensino médio frente aos diferentes registros de representação dos números, quando é analisada a propriedade da densidade. Constatou-se que apesar do envolvimento dos participantes, persistem algumas dificuldades identificadas em pesquisas anteriores, como por exemplo, a associação da representação infinita com irracionalidade e a classificação de um número racional como sendo somente aquele que tem representação finita.
	
	O subprojeto 8 referente a \textit{Modelagem como estratégia de ensino e aprendizagem}, introduzindo o conceito de modelagem, equações diferenciais, associação de fenômenos reais à objetos matemáticos. Ficou aparente a aceitação momentânea da utilização da modelagem matemática para a prática docente. No entanto, a ‘aparente aceitação’ dessa metodologia necessita de futuras comprovações.
	
	O subprojeto \textit{'transição da educação básica para o ensino superior'} (item 9) tentou associar os registros de representação à contribuição das explicitações dos conhecimentos mobilizados pelos alunos de Cálculo. O trabalho fundamentou-se na teoria dos registros de representação semiótica, destacando o papel da identificação das variáveis visuais pertinentes, no traçado de gráficos, nos tratamentos e conversões de registros e nas argumentações na língua natural. Baseou-se também no Contrato Didático, sobretudo, no que se refere a seus efeitos. A análise dos dados permitiu observar que a identificação das variáveis visuais pertinentes contribuiu para uma evolução em relação à interpretação global dos problemas propostos. O registro da língua natural, em particular, mostrou-se adequado, pois, por meio dele foram revelados conhecimentos que, geralmente, ficam “mascarados” por algoritmos mecânicos e convencionais.
	
	O subprojeto 10 ainda está em andamento, e resultados parciais apontam que o processo de transição de uma disciplina inicialmente de Análise para uma de Cálculo foi lento, gradual e com diversas idas e vindas, e que embora tendo passado a se chamar Cálculo em 1964, foi somente a partir do começo dos anos 1990 que se tornou, de fato, mais semelhante àquela estudada atualmente.
	
	Por fim, as conclusões desses subprojeto sugere que as componentes envolvidas no processo de ensino e aprendizagem do Cálculo abrem um imenso leque de questões a serem pesquisadas a fim de que se possa municiar de instrumentos que permitam examiná-las e questioná-las na tentativa de identificar os elementos que as compõem. 			
	

\chapter{Conclusão}

	Para esse resenhante, ficou claro alguns itens:
	\begin{itemize}
		\item O abismo entre o ensino médio e o ensino universitário, como podemos ver, livros clássicos consagrados como \citeaa{Bezerra1974} conforme ilustrado na \ref{fig:screenshot001} introduziam os conceitos de limites, derivadas e primitivas (operação inversa da derivação). Hoje provas abrangentes como ENEM renegam corpos de conhecimentos importante em detrimento de um 'suposto' didatismo, se contentando e dando palmas para o aluno que sabe ao menos as 4 operações. Particulamente comparando com livros estrangeiros fica claro o desmerecimento da matemática como disciplina fundamental.
	\end{itemize}

	\begin{figure}[!bh]
		\centering
		\includegraphics[width=0.7\linewidth]{figuras/screenshot001}
		\caption[Tijolão do Bezerra]{Tijolão do Bezerra}
		\label{fig:screenshot001}
	\end{figure}
	
	





\label{oculto:bibliografia}
%%---------------------------------------------------------------------
%% BIBLIOGRAFIA - Referências Bibliográficas
%%---------------------------------------------------------------------
% Referências bibliográficas
\bibliography{bibliografia/!!!bibliografia}	        % Elemento Obrigatório








\postextual

	% ----------------------------------------------------------
	% Glossário
	% ----------------------------------------------------------

	% Consulte o manual da classe abntex2 para orientações sobre o glossário.

	\ImprimirSimOuNao[Não]{             % Elemento Opcional
		\glossary
	}



	% ----------------------------------------------------------
	% Apêndices
	% ----------------------------------------------------------

	\ImprimirSimOuNao[Não]{             % Elemento Opcional
		% ---
		% Inicia os apêndices
		% ---
		\begin{apendicesenv}

		% Imprime uma página indicando o início dos apêndices
		\partapendices

		% ----------------------------------------------------------
		\chapter{Sobre esse trabalho}
		% ----------------------------------------------------------


			Esse documento foi produzido e programado usando-se \LaTeX, MikTeX, abntex2 e todo conteúdo possui links referencias clicáveis, sejam tabelas, figuras, imagens de vídeos, autores com seu respectivo registro bibliográfico.
			Informamos que o código fonte do projeto gerador desse PDF está disponível no endereço abaixo (legível também pelo QR Code abaixo): \\
			
			\qrset{link, height=4cm}
			\begin{center}
				\href{https://github.com/ChicoFigueiredo/Estacio-TCC-Estacio-Matematica}{
					\qrcode{https://github.com/ChicoFigueiredo/Estacio-TCC-Estacio-Matematica}
				}
			\end{center}
			\begin{center}
				{\tiny \url{https://github.com/ChicoFigueiredo/Estacio-TCC-Estacio-Matematica} }
			\end{center}
			


		% ----------------------------------------------------------
		%\chapter{Nullam elementum urna vel imperdiet sodales elit ipsum pharetra ligula
		%ac pretium ante justo a nulla curabitur tristique arcu eu metus}
		%% ----------------------------------------------------------
		%\lipsum[55-57]

		\end{apendicesenv}
		% ---
	}

	\label{oculto:anexos}
	% ----------------------------------------------------------
	% Anexos
	% ----------------------------------------------------------
	%
	% ---
	% Inicia os anexos
	% ---
	\ImprimirSimOuNao[Não]{             % Elemento Opcional
		\begin{anexosenv}

			% Imprime uma página indicando o início dos anexos
			\partanexos
			
			\makeatletter 
				\def\thesection{\alph{section}}
				\setcounter{chapter}{1}
			\makeatother
			
			\section{Conteúdo desse trabalho}


				Esse documento foi programado em \LaTeX, MikTeX, abntex2 e todo conteúdo possui links referenciais clicáveis, sejam tabelas, figuras, imagens de vídeos, autores com seu respectivo registro bibliográfico.
				Informamos que projeto gerador desse PDF está disponível no endereço (legível também pelo QR Code abaixo): \\
				\url{https://github.com/ChicoFigueiredo/estacio-Trab001-AASE-202004137859.git} \\
				\qrset{link, height=4cm}
				\begin{center}
					\href{https://github.com/ChicoFigueiredo/estacio-Trab001-AASE-202004137859.git}{
						\qrcode{https://github.com/ChicoFigueiredo/estacio-Trab001-AASE-202004137859.git}
					}
				\end{center}


				Apresentação no OneDrive: \url{https://1drv.ms/p/s!AgRBucATAhUblzAldnG4LGnWNV-r?e=ykIvGi} \\
				\begin{center}
					\href{https://1drv.ms/p/s!AgRBucATAhUblzAldnG4LGnWNV-r?e=ykIvGi}{
						\qrcode{https://1drv.ms/p/s!AgRBucATAhUblzAldnG4LGnWNV-r?e=ykIvGi}
					}
				\end{center}

				Vídeo no YouTube: \url{https://youtu.be/szsZ_Uuk1zk} \\
				\begin{center}
					\href{https://youtu.be/szsZ_Uuk1zk}{
						\qrcode{https://youtu.be/szsZ_Uuk1zk}
					}
				\end{center}


				\section{Códigos e Programas utilizados}

				\subsection{Programas em Python}
					programas em python

				\subsection{Programas em R}
					programas em r

				\section{Lista de questões}
					lista de questões

			
		\end{anexosenv}
	}
	
	
	%%---------------------------------------------------------------------
	%% INDICE REMISSIVO
	%%---------------------------------------------------------------------
	\ImprimirSimOuNao[Sim]{             % Elemento Opcional
		\phantompart
		\printindex
	}
	%---------------------------------------------------------------------


\end{document}

% -----------------------------------------------------------------
% Fim do Documento
% -----------------------------------------------------------------
